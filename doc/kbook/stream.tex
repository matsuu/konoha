%¥chapter{システムとコンテキスト}
%¥section{言語処理系}
%¥section{ガベージ・コレクション}
%¥section{オペレーティングシステム}
%¥subsection{バーチャル変数 os}
%¥begin{quote}
%¥begin{jverbatim}
%
%¥end{jverbatim}
%¥end{quote}
%¥section{環境変数}
%¥section{コンテキスト Context}
%¥section{プロパティ変数}

¥chapter{ストリーム}

ストリームは、バイトデータの読み書きをシーケンシャル(順序通り)に行うデータ構造である。読み書きされたバイトデータは、ファイルやソケットであり、外部プログラムとの間を送受信される。Konoha では、入力ストリームと出力ストリームをそれぞれ読み込み、書き込み専用のストリームとして分離し、InputStreamクラスとOutputStreamクラスとして提供している。

% さて、文字コードの問題が少々あらわれる。Konohaの InputStreamと OutputStreamは、C言語由来の低水準APIをメソッドをサポートしている。同時に、文字列のエンコーディングを変換しながら入出力する高水準APIもメソッドとしてサポートしている。

¥section{標準ストリーム}

Konohaは、システム定数 ¥verb|IN|, ¥verb|OUT|, ¥verb|ERR| を通して、標準入力(stdin)、標準出力(stdout), 標準出力(stderr)にそれぞれ相当する各種ストリームオブジェクトが得られる。

¥begin{quote}
¥begin{jverbatim}
>>> OUT
"/dev/stdout"
>>> OUT.println("hi");
hi
¥end{jverbatim}
¥end{quote}

また、それぞれのストリームのデフォルト値として、¥verb|"/dev/null"|に相当するストリームを持っている。これらは、何のデータも読み込めないし、何のデータも書き込めない特別なストリームである。

¥begin{quote}
¥begin{jverbatim}
>>> out = default(OutputStream);
>>> out
"/dev/null"
>>> out.println("hi");             // 何も出力されない
>>> 
¥end{jverbatim}
¥end{quote}

¥subsection{生成}

新しいストリームは、InputStreamもしくはOutputStreamクラスのコンストラクタにリソース名を与えて生成する。何らかの理由でストリームが生成できない場合は、¥verb|IO!!|例外がスローされる。

¥begin{quote}
¥begin{jverbatim}
in = new InputStream("file.txt");
foreach(String line from in) {
	print line;
}
in.close()
¥end{jverbatim}
¥end{quote}

入力ストリームの種類は、リソース名から自動的に判断される。Konoha は、下記のリソース以外にもストリームドライバーをインストールすることで、種類を拡張することができる。

¥begin{itemize}
¥item{¥bf ファイル} --- ファイルパス(ファイル名)を与えるし、もしくは明示的に¥verb|file:|タグをリソース名の先頭につける。
¥item{¥bf WEBリソース} --- URLを与える。ちなみに、これは ¥verb|http:| タグで始まるリソース名となっている。
¥end{itemize}

また、文字列やバイト列など、内部リソース(メモリ)をストリームとして抽象化する扱うこともできる。これは、マップキャスト演算子を用いて入力ストリームを得ることで生成する。

¥begin{quote}
¥begin{jverbatim}
>>> String data = "naruto";
>>> in = (InputStream)data;
>>> in.getc();


¥end{jverbatim}
¥end{quote}

注意:文字列は、UTF8でエンコーディングされたバイトストリームとして扱われる。

¥subsection{データの読み込み}

入力ストリームは、バイト列である。バイトは、Konoha では、0から255までの整数で表現される。InputStreamクラスは、C言語に由来する低水準な読み込みメソッドを備えている。

ひとつは、1バイトずつ読み込む ¥verb|getc()|メソッドである。正しくストリームから読めた場合は、0~255 の値を返すか、もしストリームの終端に達した場合は、EOF を返す。

¥begin{quote}
¥begin{jverbatim}
>>> InputStream in;
>>> while(ch = in.getc() != EOF) {
...   OUT << %c(ch);
... }
¥end{jverbatim}
¥end{quote}

もうひとつは、¥verb|read()|メソッドで、バッファ({¥sf byte[]})へ指定された指定されたバッファサイズ(¥verb|buf|)分を一度に読み込むことができる。戻り値は、実際に読み込まれたサイズであり、0からバッファサイズまでのどれかの値となる。

¥begin{quote}
¥begin{jverbatim}
>>> InputStream in;    
>>> byte[] buf = new byte[4096];
>>> while((in.read(buf) != 0) {
...   OUT << buf;
... }
¥end{jverbatim}
¥end{quote}

¥subsection{テキストの読み込み}


¥subsection{入力ストリームの終了}


¥section{{¥sf OutputStream} 出力ストリーム}

¥section{ファイル}

¥section{パイプ}

パイプは、プロセス間通信の最もシンプルで広く使われている方法です。Konoha では、UNIX(互換)パッケージを用いることで、InputStreamか OutputStreamのどちらをパイプに連結することができます。

パイプを識別するタグは、¥verb|'pipe:'|, ¥verb|'sh:'|, ¥verb|'cmd:'| のどれかです。

¥subsection{外部プログラムからデータを受ける}

パイプが最も活躍するシチュエーションは、外部のコマンド(プログラム)を起動し、その実行結果をえるときです。

次は、UNIX コマンド(ls)を実行し、その実行結果を1行ずつ表示するスクリプトの例です。Windows の場合は、ls コマンドがないので、代わりに dir で試してみましょう。

¥begin{quote}
¥begin{jverbatim}
using unix.*;
in = new InputStream("pipe:ls -l");
foreach(line from in) {
  print line;
}
in.close()
¥end{jverbatim}
¥end{quote}

¥subsection{外部プログラムへデータを送信する}

パイプでは、InputStreamの代わりに、OutputStreamを用いれば、ストリームをとしてデータの送信が可能になります。

¥begin{comment}

パイプが最も活躍するシチュエーションは、外部のコマンド(プログラム)を起動し、その実行結果をえるときです。

次は、UNIX コマンド(ls)を実行し、その実行結果を1行ずつ表示するスクリプトの例です。Windows の場合は、ls コマンドがないので、代わりに dir で試してみましょう。

¥begin{quote}
¥begin{jverbatim}
using unix.*;
in = new InputStream("pipe:ls -l");
foreach(line from in) {
  print line;
}
in.close()
¥end{jverbatim}
¥end{quote}

¥end{comment}

¥section{Socket}



