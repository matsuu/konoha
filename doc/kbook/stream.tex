%\chapter{システムとコンテキスト}
%\section{言語処理系}
%\section{ガベージ・コレクション}
%\section{オペレーティングシステム}
%\subsection{バーチャル変数 os}
%\begin{quote}
%\begin{jverbatim}
%
%\end{jverbatim}
%\end{quote}
%\section{環境変数}
%\section{コンテキスト Context}
%\section{プロパティ変数}
\chapter{ストリーム}

{\sf InputStream} は入力ストリーム、{\sf OutputStream}は出力ストリームをそれぞれバイトシーケンスとして抽象化したクラスである。ストリームは、種類によるが、原則的にメモリの上限を気にすることなく、バイトデータの読み書きが可能である。また、バイト配列({¥tt byte[]})とは異なりランダムアクセスできず、読み込み({\sf InputStream})もしくは書き込み({\sf OutputStream})操作のどちらか、
シーケンシャルにサポートするのみである。

Konohaの{\sf InputStream}と{\sf OutputStream}は、C言語由来の低水準APIをメソッドをサポートしている。同時に、文字列のエンコーディングを変換しながら入出力する高水準APIもメソッドとしてサポートしている。

%入出力ストリームは、歴史も長いよく抽象化されたデータ構造である。様々なリソースとの情報のやりとりを、バイト列のストリームとして抽象化して扱うことができる。Konoha では、C言語レベルのコンポジットパターンでストリームの種類を抽象化し、{\sf InputStream} と {\sf OutputStream}の2つのクラスから様々な種類のストリームを扱うことを可能にしている。

%として利用することができる。伝統的なオブジェクト指向クラス設計では、これらのクラスを階層構造で設計する。例えば、Java は、{\sf InputStream} クラスを抽象クラスとして、{\sf FileInputStream}, {\sf ByteArrayInputStream}, {\sf SocketInputStream}などのサブクラスでストリームの種類が実現している。

%{\sf InputStream} は入力ストリーム、{\sf OutputStream}は出力ストリームをそれぞれバイトシーケンスとして抽象化したクラスである。ストリームの種類に寄るが、原則的にメモリの上限を気にすることなく、バイト列の読み書きが可能である。また、バイト列とは異なりランダムアクセスできず、読み込み({\sf InputStream})もしくは書き込み({\sf OutputStream})操作のどちらかのみサポートすることになる。

\section{標準ストリーム}

Konohaは、システム定数 \verb|IN|, \verb|OUT|, \verb|ERR| を通して、標準入力(stdin)、標準出力(stdout), 標準出力(stderr)にそれぞれ相当する各種ストリームオブジェクトが得られる。

\begin{quote}
\begin{jverbatim}
>>> OUT
"/dev/stdout"
>>> OUT.println("hi");
hi
\end{jverbatim}
\end{quote}

また、それぞれのストリームのデフォルト値として、"/dev/null"に相当するストリームを持っている。これらは、何のデータも読み込めないし、何のデータも書き込めない特別なストリームである。

\begin{quote}
\begin{jverbatim}
>>> out = default(OutputStream);
>>> out
"/dev/null"
>>> out.println("hi");             // 何も出力されない
>>> 
\end{jverbatim}
\end{quote}

\subsection{生成}

新しいストリームは、{\sf InputStream}もしくは{\sf OutputStream}クラスのコンストラクタにリソース名を与えて生成する。何らかの理由でストリームが生成できない場合は、\verb|IO!!|例外がスローされる。

\begin{quote}
\begin{jverbatim}
in = new InputStream("file.txt");
foreach(String line from in) {
	print line;
}
in.close()
\end{jverbatim}
\end{quote}

入力ストリームの種類は、リソース名から自動的に判断される。Konoha は、下記のリソース以外にもストリームドライバーをインストールすることで、種類を拡張することができる。

\begin{itemize}
\item{\bf ファイル} --- ファイルパス(ファイル名)を与えるし、もしくは明示的に\verb|file:|タグをリソース名の先頭につける。
\item{\bf WEBリソース} --- URLを与える。ちなみに、これは \verb|http:| タグで始まるリソース名となっている。
\end{itemize}

また、文字列やバイト列など、内部リソース(メモリ)をストリームとして抽象化する扱うこともできる。これは、マップキャスト演算子を用いて入力ストリームを得ることで生成する。

\begin{quote}
\begin{jverbatim}
>>> String data = "naruto";
>>> in = (InputStream)data;
>>> in.getc();


\end{jverbatim}
\end{quote}

注意:文字列は、UTF8でエンコーディングされたバイトストリームとして扱われる。

\subsection{データの読み込み}

入力ストリームは、バイト列である。バイトは、Konoha では、0から255までの整数で表現される。{\sf InputStream} クラスは、C言語に由来する低水準な読み込みメソッドを備えている。

ひとつは、1バイトずつ読み込む \verb|getc()|メソッドである。正しくストリームから読めた場合は、0~255 の値を返すか、もしストリームの終端に達した場合は、EOF を返す。

\begin{quote}
\begin{jverbatim}
>>> InputStream in;
>>> while(ch = in.getc() != EOF) {
...   OUT << %c(ch);
... }
\end{jverbatim}
\end{quote}

もうひとつは、\verb|read()|メソッドで、バッファ({\sf byte[]})へ指定された指定されたバッファサイズ(\verb|buf|)分を一度に読み込むことができる。戻り値は、実際に読み込まれたサイズであり、0からバッファサイズまでのどれかの値となる。

\begin{quote}
\begin{jverbatim}
>>> InputStream in;    
>>> byte[] buf = new byte[4096];
>>> while((in.read(buf) != 0) {
...   OUT << buf;
... }
\end{jverbatim}
\end{quote}

\subsection{テキストの読み込み}


\subsection{入力ストリームの終了}


\section{{\sf OutputStream} 出力ストリーム}

\section{ファイル}

\section{パイプ}

パイプは、プロセス間通信の最もシンプルで広く使われている方法です。Konoha では、UNIX(互換)パッケージを用いることで、{\sf InputStream} か {\sf OutputStream} のどちらをパイプに連結することができます。

パイプを識別するタグは、\verb|'pipe:'|, \verb|'sh:'|, \verb|'cmd:'| のどれかです。

\subsection{外部プログラムからデータを受ける}

パイプが最も活躍するシチュエーションは、外部のコマンド(プログラム)を起動し、その実行結果をえるときです。

次は、UNIX コマンド(ls)を実行し、その実行結果を1行ずつ表示するスクリプトの例です。Windows の場合は、ls コマンドがないので、代わりに dir で試してみましょう。

\begin{quote}
\begin{jverbatim}
using unix.*;
in = new InputStream("pipe:ls -l");
foreach(line from in) {
  print line;
}
in.close()
\end{jverbatim}
\end{quote}

\subsection{外部プログラムへデータを送信する}

パイプでは、{\sf InputStream} の代わりに、{\sf OutputStream}を用いれば、ストリームをとしてデータの送信が可能になります。

\begin{comment}

パイプが最も活躍するシチュエーションは、外部のコマンド(プログラム)を起動し、その実行結果をえるときです。

次は、UNIX コマンド(ls)を実行し、その実行結果を1行ずつ表示するスクリプトの例です。Windows の場合は、ls コマンドがないので、代わりに dir で試してみましょう。

\begin{quote}
\begin{jverbatim}
using unix.*;
in = new InputStream("pipe:ls -l");
foreach(line from in) {
  print line;
}
in.close()
\end{jverbatim}
\end{quote}

\end{comment}

\section{Socket}



